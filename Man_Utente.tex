\documentclass[fontsize=12pt,a4paper,DIV=16]{scrreprt}

\usepackage{sty/preamble}
\usepackage{sty/preamble-math}
\usepackage{sty/preamble-code}

\import{./tex/utente}{title}
\author{
	\\Iuri Antico \textsl{matricola}:
	\texttt{753144}
	\and \\
	Beatrice Balzarini \textsl{matricola}:
	\texttt{752257}
	\and \\
	Michael Bernasconi \textsl{matricola}:
	\texttt{752259}
	\and \\
	Gabriele Borgia \textsl{matricola}:
	\texttt{753262}\\\\
}

\begin{document}
	\maketitle
	\tableofcontents
	\chapter{Il Programma}
		\section{Introduzione}
		\textsl{\textbf{Climate Monitoring}} \`e un'applicazione finalizzata alla rilevazione e all'analisi di parametri climatici fornita da centri di monitoraggio sul territorio italiano, in grado di rendere disponibili, a operatori ambientali e comuni cittadini, i dati relativi alla propria zona di interesse.
			\subsection{Funzionamento generale dell'applicazione}
			L'applicazione offre diverse funzionalit\`a tra cui:
			\begin{itemize}
				\item la creazione di nuovi centri di monitoraggio climatico;
				\item l'inserimento di nuove misurazioni con i relativi parametri climatici;
				\item la visualizzazione delle aree geografiche presenti nel database;
				\item la visualizzazione di centri di monitoraggio presenti nel database;
				\item la visualizzazione delle misurazioni presenti nel database;
				\item la visualizzazione degli operatori registrati nel database.
			\end{itemize}
			\pagebreak
			I parametri climatici che possono essere rilevati e inseriti sono:
			\begin{itemize}
				\item \textsl{Vento}, la cui velocit\`a \`e espressa in km/h;
				\item \textsl{Umidit\`a}, misurata in percentuale (\%);
				\item \textsl{Pressione}, misurata in hPa;
				\item \textsl{Temperatura}, misurata in C°;
				\item \textsl{Precipitazione}, misurata in mm di pioggia;
				\item \textsl{Altitudine dei ghiacciai}, misurata in m;
				\item \textsl{Massa dei ghiacciai}, misurata in kg.
			\end{itemize}
			L'intensit\`a di ogni fenomeno climatico viene misurata su una scala che va da \textbf{1} (\textsl{critico}) a \textbf{5} (\textsl{ottimale}).
			
			Possono inoltre essere presenti delle note testuali (di massimo 256 caratteri) per descrivere con più precisione i dati geografici.
			
			L'applicazione permette:
			\begin{itemize}
				\item ai \textbf{\textsl{comuni cittadini}}, di visualizzare i parametri climatici di proprio interesse relativi a ciascuna area geografica;
				\item a \textbf{\textsl{operatori autorizzati}}, di gestire le aree di interesse inserendo i vari parametri climatici.
			\end{itemize}
			Inoltre questi ultimi hanno la possibilit\`a di registrarsi, creare nuovi centri di monitoraggio e aggiungervi aree di interesse.
			
			\pagebreak
			
	\chapter{Uso dell'applicazione}
		\section{Requisiti minimi\index{Requisiti minimi}}
		Per eseguire l’applicazione \`e necessario avere a disposizione:
		\begin{itemize}
			\item Java JDK 17 o superiore;
			\item un sistema operativo a 64 bit;
			\item un terminale.
		\end{itemize}
	\chapter{Limiti della soluzione sviluppata}	
	\nocite{IuriTex}
	\bibliographystyle{alpha}
	\bibliography{bib/biblio}
\end{document}
