\maketitle

\cleardoublepage
\phantomsection
\addcontentsline{toc}{chapter}{Indice}
\tableofcontents

\listoffigures

\chapter{Il Programma}
\section{Introduzione}
\textsl{\textbf{Climate Monitoring}} \`e un'applicazione finalizzata alla rilevazione e all'analisi di parametri climatici fornita da centri di monitoraggio sul territorio italiano, in grado di rendere disponibili, a operatori ambientali e comuni cittadini, i dati relativi alla propria zona di interesse.
\subsection{Funzionamento generale dell'applicazione}
L'applicazione offre diverse funzionalit\`a tra cui:
\begin{itemize}
	\item la creazione di nuovi centri di monitoraggio climatico;
	\item l'inserimento di nuove misurazioni con i relativi parametri climatici;
	\item la visualizzazione delle aree geografiche presenti nel database;
	\item la visualizzazione di centri di monitoraggio presenti nel database;
	\item la visualizzazione delle misurazioni presenti nel database;
	\item la visualizzazione degli operatori registrati nel database.
\end{itemize}
\pagebreak
I parametri climatici che possono essere rilevati e inseriti sono:
\begin{itemize}
	\item \textsl{Vento}, la cui velocit\`a \`e espressa in km/h;
	\item \textsl{Umidit\`a}, misurata in percentuale (\%);
	\item \textsl{Pressione}, misurata in hPa;
	\item \textsl{Temperatura}, misurata in C°;
	\item \textsl{Precipitazione}, misurata in mm di pioggia;
	\item \textsl{Altitudine dei ghiacciai}, misurata in m;
	\item \textsl{Massa dei ghiacciai}, misurata in kg.
\end{itemize}
L'intensit\`a di ogni fenomeno climatico viene misurata su una scala che va da \textbf{1} (\textsl{critico}) a \textbf{5} (\textsl{ottimale}).

Possono inoltre essere presenti delle note testuali (di massimo 256 caratteri) per descrivere con più precisione i dati geografici.

L'applicazione permette:
\begin{itemize}
	\item ai \textbf{\textsl{comuni cittadini}}, di visualizzare i parametri climatici di proprio interesse relativi a ciascuna area geografica;
	\item a \textbf{\textsl{operatori autorizzati}}, di gestire le aree di interesse inserendo i vari parametri climatici.
\end{itemize}
Inoltre questi ultimi hanno la possibilit\`a di registrarsi, creare nuovi centri di monitoraggio e aggiungervi aree di interesse.

\pagebreak

\chapter{Installazione}
\section{Requisiti di sistema}
Per eseguire l’applicazione \`e necessario avere a disposizione:
\begin{itemize}
	\item Java JDK 17 o superiore;
	\item un sistema operativo a 64 bit;
	\item un terminale.
	\item server PostgreSQL
\end{itemize}

\section{Setup Ambiente}

Il client richiede il server del progetto e un database PostgreSQL.

Consultare il manuale tecnico per ulteriori informazioni su come impostare l'ambiente.

\section{Installazione programma}

Per installare il programma:

\begin{enumerate}
	\item Scaricare da Github releases (\url{https://github.com/Qu4draetto-A3B/UniLab-client/releases}) il file adatto al proprio sistema operativo.
	\item Spostarlo dove si preferisce.
	\item Fare doppio click sul file.
\end{enumerate}

\chapter{Esecuzione e uso delle funzionalità}
\section{Lancio del programma}
All’avvio dell’applicazione compare la seguente schermata, dove è possibile visualizzare il menù principale del programma.
\section{Schermata Home}
Come si può notare dall'immagine, la schermata di avvio dell'applicazione è dotata di tre pulsanti:
\begin{itemize}
	\item \textbf{\textit{Visualizza area}}, permette di visualizzare tutte le aree geografiche presenti con le relative misurazioni;
	\item \textbf{\textit{Registrazione}}, permette all'operatore di registrarsi nel sistema e accedere a una serie di funzioni a lui riservate;
	\item \textbf{\textit{Login}}: permette all'operatore già registrato di accedere al sistema. 
\end{itemize}
In tutte le schermate dell'applicazione sono presenti inoltre due pulsanti:
\begin{itemize}
	\item \textbf{freccia a sinistra}, permette di tornare alla schermata precedente;
	\item \textbf{icona della casa}, permette di tornare alla schermata Home dell'applicazione.
\end{itemize}
\subsection{Visualizza area}\label{VisualizzaArea}
Questa pagina presenta una barra di ricerca, dove è possibile effettuare delle ricerche sulle aree geografiche presenti nel database.
Con un doppio click o selezionando l'area geografica con il mouse e premendo invio, è possibile visualizzare tutte le misurazioni relative a un'area specifica.

\subsection{Registrazione}
In questa schermata sono presenti tutti i campi che l'operatore deve compilare per la registrazione. Per la conferma è presente il pulsante \textbf{Registrazione}.

In particolare, i dati che l'operatore deve inserire sono:
\begin{itemize}
	\item  \textbf{UserID}, codice identificativo dell'utente;
	\item  \textbf{Nome};
	\item  \textbf{Cognome};
	\item  \textbf{Codice Fiscale};
	\item  \textbf{Email};
	\item  \textbf{Password}, che verrà richiesta in caso di accessi futuri.
\end{itemize}
Una volta premuto il tasto \textbf{Registrazione} viene aperta una nuova schermata, che ha lo scopo di associare un centro di monitoraggio all'operatore appena registrato.

In particolare l'operatore deve scegliere se creare un nuovo centro di monitoraggio o selezionarne uno già esistente, premendo il pulsante corrispondente.
\subsection{Centro di monitoraggio esistente}
\subsection{Nuovo centro di monitoraggio} \label{NuovoCentroDiMonitoraggio}
In caso di selezione di quest'ultimo viene aperta la seguente pagina, nella quale andranno inseriti i dati per creare un nuovo centro di monitoraggio:
\begin{itemize}
	\item \textbf{Città}
	\item \textbf{Provincia}
	\item \textbf{Via}
	\item \textbf{Numero civico}
	\item \textbf{CAP}
\end{itemize}
Per la conferma è presente il relativo pulsante.
\subsection{Login}
Nella seguente schermata sono presenti i campi per l'inserimento dei dati per l'accesso dell'operatore.

In particolare troviamo:
\begin{itemize}
	\item \textbf{UserID} 
	\item \textbf{Password}
\end{itemize} 

Per la conferma è presente il pulsante \textbf{Login}.
\subsection{Inserisci Parametri}

\subsection{Crea nuovo centro di monitoraggio}
Vedi \ref{NuovoCentroDiMonitoraggio} 
\subsection{Visualizza Aree}
Vedi \ref{VisualizzaArea}

\section{Esci}
Per uscire dall'applicazione è sufficiente chiudere la finestra, cliccando sull'icona a croce ("\textit{Chiudi}" o "\textit{Close}") in alto a destra:

\chapter{Limiti della soluzione sviluppata}
\begin{itemize}
	\item l'istanza del database deve essere l'unica istanza attiva di PostgreSQL, altrimenti il server rischia di connettersi all'istanza sbagliata
\end{itemize}

\nocite{IuriTex}
\bibliographystyle{alpha}
\bibliography{bib/biblio}
