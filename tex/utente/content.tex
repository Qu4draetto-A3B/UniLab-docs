\maketitle

\cleardoublepage
\phantomsection
\addcontentsline{toc}{chapter}{Indice}
\tableofcontents

\listoffigures
\listoftables
\lstlistoflistings

\chapter{Il Programma}
\section{Introduzione}
\textsl{\textbf{Climate Monitoring}} \`e un'applicazione finalizzata alla rilevazione e all'analisi di parametri climatici fornita da centri di monitoraggio sul territorio italiano, in grado di rendere disponibili, a operatori ambientali e comuni cittadini, i dati relativi alla propria zona di interesse.
\subsection{Funzionamento generale dell'applicazione}
L'applicazione offre diverse funzionalit\`a tra cui:
\begin{itemize}
	\item la creazione di nuovi centri di monitoraggio climatico;
	\item l'inserimento di nuove misurazioni con i relativi parametri climatici;
	\item la visualizzazione delle aree geografiche presenti nel database;
	\item la visualizzazione di centri di monitoraggio presenti nel database;
	\item la visualizzazione delle misurazioni presenti nel database;
	\item la visualizzazione degli operatori registrati nel database.
\end{itemize}
\pagebreak
I parametri climatici che possono essere rilevati e inseriti sono:
\begin{itemize}
	\item \textsl{Vento}, la cui velocit\`a \`e espressa in km/h;
	\item \textsl{Umidit\`a}, misurata in percentuale (\%);
	\item \textsl{Pressione}, misurata in hPa;
	\item \textsl{Temperatura}, misurata in C°;
	\item \textsl{Precipitazione}, misurata in mm di pioggia;
	\item \textsl{Altitudine dei ghiacciai}, misurata in m;
	\item \textsl{Massa dei ghiacciai}, misurata in kg.
\end{itemize}
L'intensit\`a di ogni fenomeno climatico viene misurata su una scala che va da \textbf{1} (\textsl{critico}) a \textbf{5} (\textsl{ottimale}).

Possono inoltre essere presenti delle note testuali (di massimo 256 caratteri) per descrivere con più precisione i dati geografici.

L'applicazione permette:
\begin{itemize}
	\item ai \textbf{\textsl{comuni cittadini}}, di visualizzare i parametri climatici di proprio interesse relativi a ciascuna area geografica;
	\item a \textbf{\textsl{operatori autorizzati}}, di gestire le aree di interesse inserendo i vari parametri climatici.
\end{itemize}
Inoltre questi ultimi hanno la possibilit\`a di registrarsi, creare nuovi centri di monitoraggio e aggiungervi aree di interesse.

\pagebreak

\chapter{Uso dell'applicazione}
\section{Requisiti minimi\index{Requisiti minimi}}
Per eseguire l’applicazione \`e necessario avere a disposizione:
\begin{itemize}
	\item Java JDK 17 o superiore;
	\item un sistema operativo a 64 bit;
	\item un terminale.
\end{itemize}
\chapter{Limiti della soluzione sviluppata}
\chapter{Avviare l'applicazione}
All’avvio dell’applicazione compare la seguente schermata, dove è possibile visualizzare il menu principale del programma.
\chapter{Menù principale}
\section{Schermata Home}
Questa è la schermata di avvio dell'applicazione, sono presente tre pulsanti:
\begin{itemize}
	\item \textbf{Visualizza area:} permette di visualizzare tutte le aree geografiche presenti con le relative misurazioni 
	\item \textbf{Registrazione:} permette all'operatore di registrarsi
	\item \textbf{Login:} permette all'operatore di accedere alla propria area personale. 
\end{itemize}
In tutte le schermate successive a quella Home sono presenti due pulsanti:
\begin{itemize}
	\item \textbf{freccia:} che permette di tornare alla schermata precedente  
	\item \textbf{casa:} che permette di tornare alla schermata home dell'applicazione
\end{itemize}
\subsection{Visualizza area}
In questa pagina è presenta una barra di ricerca, dove è possibile cercare le aree geografiche presenti nel database. Con un doppio click o selezionando l'area geografica con il mouse e premendo invio, è possibile visualizzare tutte le misurazioni per quella specifica area.

\subsection{Registrazione}\label{VisualizzaArea}
In questa schermata dell'applicazione sono presenti tutti i campi che l'operatore deve riempire per rendere la registrazione possibile: inoltre è presente il tasto registrazione per confermare inserimento dei parametri.
I parametri che l'operatore deve inserire sono:
\begin{itemize}
	\item  \textbf{UserID}
	\item  \textbf{Nome}
	\item  \textbf{Cognome}
	\item  \textbf{Codice Fiscale}
	\item  \textbf{Email}
	\item  \textbf{Password}
\end{itemize}
Una volta che viene premuto il tasto \textbf{Registrazione} si apre una nuova schermata per associare il centro di monitoraggio all'operatore.
In questa schermata sono presenti due bottoni in cui l'operatore deve scegliere se creare un nuovo centro di monitoraggio o sceglierne uno già esistente.
\subsection{Centro di monitoraggio esistente}
\subsection{Nuovo centro di monitoraggio} \label{NuovoCentroDiMonitoraggio}
Nella seguente pagina sono presenti i campi per creare un nuovo centro di monitoraggio:
\begin{itemize}
	\item \textbf{Città}
	\item \textbf{Provincia}
	\item \textbf{Strada}
	\item \textbf{Numero civico}
	\item \textbf{Zipcode}
\end{itemize}
Oltre ai campi è presente il buttone \textbf{conferma} per validare l'operazione di creazione del nuovo centro di monitoraggio
\section{Login}
Nella seguente pagina sono presenti i campi che l'operatore deve inserire per accedere alle proprie funzioni. I campi sono:
\begin{itemize}
	\item \textbf{UserID} 
	\item \textbf{Password}
\end{itemize} 
\textbf{UserID} e \textbf{Password} sono stati precedentemente inserirti durante la registrazione.
Inoltre la pagina contiene il buttone \textbf{Login} per accedere.
\subsection{Inserisci Parametri}

\subsection{Crea nuovo centro di monitoraggio}
vedi :\ref{NuovoCentroDiMonitoraggio} 	\textbf{[Centro di monitoraggio esistente]}
\subsection{Visualizza Aree}
vedi: \ref{VisualizzaArea} 	\textbf{[Visualizza Area]}
\subsection{Impostazioni}

\section{Esci}
Per uscire dall'applicazione è sufficiente chiudere la finestra, cliccando sull'icona a croce ("\textit{Chiudi}" o "\textit{Close}") in alto a destra:


\nocite{IuriTex}
\bibliographystyle{alpha}
\bibliography{bib/biblio}
